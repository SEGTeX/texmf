\documentclass{geophysics}

% An example of defining macros
\newcommand{\rs}[1]{\mathstrut\mbox{\scriptsize\rm #1}}
\newcommand{\rr}[1]{\mbox{\rm #1}}

\begin{document}

\title{An example SEG paper template,\\with a two-line title}

\author{Joe Dellinger}
\address{BP Amoco UTG, 200 Westlake Park Blvd, Houston, TX, 77079, USA}

%\author{Another Author}
%\address{SEG business office, Tulsa, OK}

\date{\today}

\maketitle

{Short title: example SEG paper}

\begin{abstract}
The abstract goes here.
This new distribution of SEGTeX contains a new bibliography database file.
It also includes some minor changes to conform to new SEG submission standards.
\end{abstract}

\section{INTRODUCTION}
Here are some examples of how to do citations using SEG.bib.
The SEG.bib file contains the entire SEG database up to mid-2000. It was
put together by Ossie Moore and Sergey Fomel. Please read the terms and
conditions at the start of the SEG.bib file before using it or
redistributing it.

Alford \shortcite{SEG-1986-S9.6} rotation is a widely used technique for
separating split shear waves in $2{\times}2$-component
seismic data~\cite{GEO53-03-03040313}.
More recently it has come to be used for processing $2{\times}2$-C
crossed-dipole data as well \cite{SEG-1994-0233,SEG-1994-1139,SEG-1994-1143}.
See Figure~\ref{fig:waves}.

\plot{waves}{width=6.0in}{You can keep figures from getting
  stretched by setting only one dimension.}

\subsection{Here is a subsection}
Section headings should be in all caps. Subsection headings should only
have the first letter of the first word capitalized.
See table~\ref{rotation} for an example of a table.

Here are examples of equations involving vectors and tensors:
\begin{equation}
\tensor{R} = \pmatrix{R_{\rs{XX}} & R_{\rs{YX}} \cr R_{\rs{XY}} & R_{\rs{YY}}} =
\tensor{P}_{M\rightarrow R} \; \tensor{D} \; \tensor{P}_{S\rightarrow M}
\;\;\; \tensor{S} \ \ \  ,
\label{SVD}
\end{equation}
and
\begin{equation}
R_{j,m}(\omega) =
\sum_{n=1}^{N} \, \,
P_{j}^{(n)}(\vector{x}_R) \, \,
D^{(n)}(\omega) \, \,
P_{m}^{(n)}(\vector{x}_S) \ \ \ .
\label{SVDray}
\end{equation}
Note that the macros for the ``vector'' and ``tensor'' commands have been
changed to force tensors to be bold uppercase and vectors to be bold lowercase,
in compliance with current SEG submission standards. This is so that documents
typeset to the old standards will print out according to the new ones:
e.g., tensor $\tensor{t}$ (note converted to uppercase) and
vector $\vector{V}$ (note converted to lowercase).

%
% Note a tilde makes an "unbreakable space".
%
Here is an example of referring to equations (\ref{SVD}) and~(\ref{SVDray}).

\section{CONCLUSIONS}
I conclude that
someone needs to come up with a \LaTeX2e\ set of macros for the SEG to use!

\section{ACKNOWLEDGEMENTS}
First, thanks to Sergey Fomel for creating the new SEG.bib database!

I wish to thank Ivan P\v{s}en\v{c}\'{\i}k and
Fr\'ed\'eric Billette for having names with non-English letters in them.
I wish to thank
\v{C}erven\'{y}~\shortcite{Cerveny} for
providing an example of how to make a bib file that includes an author
whose name begins with a non-English character, and
Forgues~\shortcite{forgues96} for providing both an example of referencing a
Ph.D. thesis and yet more non-English characters.

\bibliographystyle{segnat}
\bibliography{SEG,books}

\begin{table}
$$
\pmatrix{
16.0398 & 9.30334 & -3.01053 & 7.32306 & 0.187091 & -.0887023 \cr
9.30334 & 16.0000 & 3.26742 & 0. & -.641486 & 0. \cr
-3.01053 & 3.26742 & 16.0398 & -.00932298 & -.187091 & 0.857407 \cr
7.32306 & 0. & -.00932298 & 6.36847 & -.340491 & 0.299948 \cr
0.187091 & -.641486 & -.187091 & -.340491 & 10.3656 & -4.07418 \cr
-.0887023 & 0. & 0.857407 & 0.299948 & -4.07418 & 3.54618 \cr
}
$$
\caption{Here is an example of how to do a table.}
\label{rotation}
\end{table}

\end{document}
